\documentclass[10pt,a4paper]{article}

\usepackage[utf8]{inputenc}
\usepackage[spanish]{babel}
\usepackage{amsmath}
\usepackage{amsfonts}
\usepackage{amssymb}
\usepackage{geometry}
\usepackage{enumitem}
\usepackage{titlesec}
\geometry{margin=2cm}

\titleformat{\section}{\Large\bfseries}{\thesection}{1em}{}
\titleformat{\subsection}{\large\bfseries}{\thesubsection}{1em}{}

\setlength{\parindent}{0pt}
\setlength{\parskip}{0.3em}

\begin{document}

\begin{center}
\textbf{\LARGE FORMULARIO DE MECÁNICA Y ONDAS I}
\end{center}

\hrulefill

% ==========================================================
\section{SISTEMAS DE REFERENCIA NO INERCIALES Y ROTACIÓN}

\subsection*{GLOSARIO DE VARIABLES}
\begin{itemize}[noitemsep]
    \item $\vec r,\vec r'$ = posición
    \item $\vec v,\vec v'$ = velocidad
    \item $\vec a,\vec a'$ = aceleración
    \item $\vec\Omega$ = velocidad angular del sistema (rad/s), $\omega = |\vec\Omega|$
    \item $\dot{\vec\Omega}$ = aceleración angular (rad/s$^2$)
    \item $r,R$ = radios (m)
    \item $g$ = gravedad ($9.8$ m/s$^2$)
    \item $\vec\Omega_T$ = rotación de la Tierra (rad/s)
    \item $\lambda$ = latitud
    \item $\alpha$ = ángulo de tiro
    \item $v_0$ = velocidad inicial
    \item $d$ = desviación lateral
    \item $t_c$ = tiempo de caída corregido
    \item $R_0$ = alcance sin rotación, $\Delta R$ = corrección del alcance
\end{itemize}

\subsection*{FÓRMULAS}

\textbf{Derivada en sistema en rotación:}
\begin{align*}
\left(\frac{d\vec A}{dt}\right)_{\!in} &= \left(\frac{d\vec A}{dt}\right)_{\!rot} + \vec\Omega \times \vec A
\end{align*}

\textbf{Velocidad y aceleración (inercial vs. rotante):}
\begin{align*}
\vec v &= \vec v' + \vec\Omega \times \vec r' + \vec v_{orig} \\
\vec a &= \vec a' + \dot{\vec\Omega}\times \vec r' + \vec\Omega\times(\vec\Omega\times \vec r') + 2\vec\Omega\times \vec v' + \vec a_{orig}
\end{align*}

\textbf{Aceleraciones ficticias:}
\begin{align*}
\vec a_{cor} &= 2\vec\Omega \times \vec v' \\
\vec a_{cf} &= \vec\Omega \times (\vec\Omega \times \vec r')
\end{align*}

\textbf{Ecuación de movimiento en el sistema no inercial:}
\begin{align*}
m\vec a_{rot} &= \sum \vec F_{\text{reales}} + \vec F_{\text{fict}}
\end{align*}

\textbf{Fuerzas ficticias (forma estándar):}
\begin{align*}
\vec F_C &= -2m\,\vec\Omega \times \vec v_{rot} \\
\vec F_{cf} &= -m\,\vec\Omega \times (\vec\Omega \times \vec r) \\
\vec F_E &= -m\,\dot{\vec\Omega} \times \vec r
\end{align*}

\textbf{Movimiento circular (cinemática):}
\begin{align*}
v &= \omega r \\
a_c &= \omega^2 r = \frac{v^2}{r}
\end{align*}

\textbf{Gravedad artificial en anillo:}
\begin{align*}
\omega^2 R &= g
\quad \Rightarrow \quad
\omega = \sqrt{\frac{g}{R}}, \qquad v = \omega R = \sqrt{gR}
\end{align*}

\textbf{Superficie de un líquido en rotación:}
\begin{align*}
z(r) &= \frac{\omega^2}{2g}\,r^2 + \text{cte}
\end{align*}

\textbf{Coriolis en la Tierra (aprox.):}
\begin{align*}
\vec a_C &= -2\vec\Omega_T \times \vec v
\end{align*}

\textbf{Desviación lateral (resultado dado):}
\begin{align*}
d &= \frac{4v_0^3}{g^2}\,\omega\,\sin\lambda\,\sin^2\alpha\,\cos\alpha
\end{align*}

\textbf{Tiempo de caída corregido (resultado dado):}
\begin{align*}
t_c &= \frac{2v_0\sin\alpha}{g - 2\omega v_0\cos\alpha\cos\lambda}
\end{align*}

\textbf{Aproximación:}
\begin{align*}
(1-x)^{-1} &\approx 1+x \qquad (|x|\ll 1)
\end{align*}

\textbf{Corrección del alcance (resultado dado):}
\begin{align*}
\Delta R &= \sqrt{\frac{2R_0^3}{g}}\;\omega\cos\lambda
\left(\cot^{1/2}\alpha - \frac{1}{3}\tan^{3/2}\alpha\right)
\end{align*}

\hrulefill

% ==========================================================
\section{SISTEMAS DE PARTÍCULAS, CHOQUES Y DISPERSIÓN}

\subsection*{GLOSARIO DE VARIABLES}
\begin{itemize}[noitemsep]
    \item $m_i$ = masa de la partícula $i$, $M=\sum_i m_i$ = masa total
    \item $\vec r_i,\vec v_i$ = posición/velocidad de la partícula $i$
    \item $\vec R_{CM},\vec V_{CM},\vec A_{CM}$ = posición/velocidad/aceleración del CM
    \item $\vec P$ = momento lineal total
    \item $\vec L$ = momento angular, $\vec\tau$ = torque
    \item $\vec F_{ext}$ = fuerza externa total
    \item $\vec J$ = impulso
    \item $k$ = constante del muelle, $x_{max}$ = compresión máxima
    \item $N_0$ = incidentes, $n$ = densidad numérica, $t$ = espesor
    \item $\sigma$ = sección eficaz, $\frac{d\sigma}{d\Omega}$ = sección diferencial
    \item $b$ = parámetro de impacto, $\theta,\varphi$ = ángulos, $d\Omega$ = ángulo sólido
\end{itemize}

\subsection*{FÓRMULAS}

\textbf{Centro de masas:}
\begin{align*}
\vec R_{CM} &= \frac{1}{M}\sum_i m_i\vec r_i,
\qquad
M = \sum_i m_i \\
\vec V_{CM} &= \dot{\vec R}_{CM} = \frac{1}{M}\sum_i m_i\vec v_i
\end{align*}

\textbf{Momento lineal total y 2ª ley para el sistema:}
\begin{align*}
\vec P &= \sum_i m_i\vec v_i = M\vec V_{CM} \\
\sum \vec F_{ext} &= \frac{d\vec P}{dt} = M\vec A_{CM}
\end{align*}

\textbf{Momento angular y torque:}
\begin{align*}
\vec L &= \sum_i \vec r_i \times (m_i\vec v_i) \\
\vec\tau &= \frac{d\vec L}{dt}, \qquad \vec\tau_{ext} = \frac{d\vec L}{dt}
\end{align*}

\textbf{Descomposición del momento angular:}
\begin{align*}
\vec L &= \vec L_{CM} + \vec R_{CM}\times \vec P \\
\vec L &= \vec R_{CM}\times \vec P + \sum_i \vec r_i' \times \vec p_i'
\qquad
(\vec r_i'=\vec r_i-\vec R_{CM})
\end{align*}

\textbf{Energía cinética (descomposición):}
\begin{align*}
K &= \frac12 M V_{CM}^2 + K_{int}, \qquad
K_{int} = \sum_i \frac12 m_i|\vec v_i - \vec V_{CM}|^2
\end{align*}

\textbf{Impulso:}
\begin{align*}
\vec J &= \int \vec F\,dt = \Delta \vec P
\end{align*}

\textbf{Choques:}
\begin{align*}
\sum \vec p_i &= \sum \vec p_f
\end{align*}
Elástico: además $K_i = K_f$. \\
Inelástico perfecto (1D):
\begin{align*}
v_f &= \frac{m_1 v_1 + m_2 v_2}{m_1+m_2}
\end{align*}

\textbf{Proyectil que comprime un muelle (esquema típico):}
\begin{align*}
V &= \frac{mv}{M+m} \\
\frac12(M+m)V^2 &= \frac12 kx_{max}^2
\quad \Rightarrow \quad
x_{max} = V\sqrt{\frac{M+m}{k}}
\end{align*}

\textbf{Dispersión (scattering):}
\begin{align*}
N_{disp} &\approx N_0\,n\,t\,\sigma \\
\frac{d\sigma}{d\Omega} &= \frac{b}{\sin\theta}\left|\frac{db}{d\theta}\right| \\
\sigma &= \int \frac{d\sigma}{d\Omega}\,d\Omega \\
d\Omega &= \sin\theta\,d\theta\,d\varphi
\end{align*}

\hrulefill

% ==========================================================
\section{SÓLIDO RÍGIDO: MASA, CM, INERCIA Y TORQUE}

\subsection*{GLOSARIO DE VARIABLES}
\begin{itemize}[noitemsep]
    \item $\lambda,\sigma,\rho$ = densidades lineal/superficial/volumétrica
    \item $M$ = masa total, $\vec R_{CM}$ = centro de masas
    \item $\mathbf I$ = tensor de inercia, $I_{ij}$ = componentes
    \item $\delta_{ij}$ = delta de Kronecker
    \item $\vec\omega$ = velocidad angular, $\vec L$ = momento angular
    \item $\vec\tau$ = torque, $T_{rot}$ = energía cinética de rotación
    \item $R,L$ = radio/longitud, $a$ = aceleración, $\alpha$ = aceleración angular
    \item $T$ = tensión
\end{itemize}

\subsection*{FÓRMULAS}

\textbf{Elementos de masa y masa total:}
\begin{align*}
dm &= \lambda\,dl, \qquad dm = \sigma\,dA, \qquad dm = \rho\,dV \\
M &= \int dm
\end{align*}

\textbf{Centro de masas (continuo):}
\begin{align*}
\vec R_{CM} &= \frac{1}{M}\int \vec r\,dm
\end{align*}

\textbf{Tensor de inercia (componentes):}
\begin{align*}
I_{xx} &= \int (y^2+z^2)\,dm, \quad
I_{yy} = \int (x^2+z^2)\,dm, \quad
I_{zz} = \int (x^2+y^2)\,dm \\
I_{xy}=I_{yx} &= -\int xy\,dm, \quad
I_{xz} = -\int xz\,dm, \quad
I_{yz} = -\int yz\,dm
\end{align*}

\textbf{Tensor de inercia (forma compacta con densidad):}
\begin{align*}
I_{ij} &= \int \rho\,(\delta_{ij}r^2 - x_i x_j)\,dV
\end{align*}

\textbf{Relación compacta:}
\begin{align*}
\vec L &= \mathbf I \cdot \vec\omega
\end{align*}

\textbf{Identidad (lámina en $x,y$):}
\begin{align*}
I_{xx}+I_{yy} &= I_{zz}
\end{align*}

\textbf{Energía cinética de rotación:}
\begin{align*}
T_{rot} &= \frac12\,\vec\omega\cdot \vec L
= \frac12\,\vec\omega^{\,T}\mathbf I\,\vec\omega
\end{align*}

\textbf{Torque:}
\begin{align*}
\vec\tau &= \frac{d\vec L}{dt}
\end{align*}
Si $\vec\omega$ constante y $\vec L \not\parallel \vec\omega$:
\begin{align*}
\vec\tau &= \vec\omega \times \vec L
= \vec\omega \times (\mathbf I\cdot\vec\omega)
\end{align*}

\textbf{Cilindro con cuerda + masa colgante:}
\begin{align*}
a &= \alpha R \\
mg - T &= ma \\
\tau &= TR = I\alpha
\end{align*}

\textbf{Momentos de inercia típicos:}
\begin{align*}
I_{\text{barra (centro)}} &= \frac{1}{12}ML^2, \qquad
I_{\text{barra (extremo)}} = \frac{1}{3}ML^2 \\
I_{\text{disco/cilindro macizo}} &= \frac12 MR^2, \qquad
I_{\text{anillo/cilindro hueco}} = MR^2 \\
I_{\text{esfera maciza}} &= \frac{2}{5}MR^2, \qquad
I_{\text{esfera hueca}} = \frac{2}{3}MR^2
\end{align*}

\hrulefill

% ==========================================================
\section{OSCILACIONES: MAS, AMORTIGUADO Y FORZADO}

\subsection*{GLOSARIO DE VARIABLES}
\begin{itemize}[noitemsep]
    \item $x$ = desplazamiento, $A$ = amplitud, $\phi$ = fase
    \item $\omega$ = frecuencia angular, $\omega_0$ = frecuencia natural
    \item $T$ = periodo, $f$ = frecuencia
    \item $k$ = constante elástica, $m$ = masa
    \item $b$ = rozamiento viscoso, $\beta$ = amortiguamiento
    \item $F_0$ = amplitud de la fuerza externa, $\Omega$ = frecuencia externa
    \item $\rho$ = densidad del fluido, $A$ (sección) = área de sección
    \item $U(x)$ = potencial
\end{itemize}

\subsection*{FÓRMULAS}

\textbf{Movimiento armónico simple (MAS):}
\begin{align*}
x(t) &= A\cos(\omega t+\phi) \quad \text{o} \quad x(t) = A\sin(\omega t+\phi) \\
\omega &= \frac{2\pi}{T}, \qquad f=\frac{1}{T}, \qquad \omega = \sqrt{\frac{k}{m}}
\end{align*}
Si $x=C\sin(\omega t)+D\cos(\omega t)$:
\begin{align*}
A &= \sqrt{C^2 + D^2}
\end{align*}

\textbf{Masa que se duplica (misma $k$):}
\begin{align*}
\omega' &= \sqrt{\frac{k}{2m}}
\end{align*}

\textbf{Oscilación por empuje (flotación):}
\begin{align*}
\Delta E &= \rho g A\,x \\
m\ddot x + (\rho g A)x &= 0 \\
\omega &= \sqrt{\frac{\rho g A}{m}}, \qquad
T = 2\pi\sqrt{\frac{m}{\rho g A}}
\end{align*}

\textbf{Oscilador amortiguado (viscoso):}
\begin{align*}
F_{roz} &= -b\dot x \\
m\ddot x + b\dot x + kx &= 0 \\
\omega_0 &= \sqrt{\frac{k}{m}}, \qquad \beta=\frac{b}{2m} \\
\omega &= \sqrt{\omega_0^2-\beta^2} \quad (\text{subamortiguado}) \\
\beta &= \omega_0 \quad \Leftrightarrow \quad b_c=2m\omega_0=2\sqrt{km} \quad (\text{crítico}) \\
\delta &= \ln\!\left(\frac{A_n}{A_{n+1}}\right) = \frac{2\pi\beta}{\omega}
\end{align*}

\textbf{Forzado:}
\begin{align*}
m\ddot x + b\dot x + kx &= F_0\cos(\Omega t) \\
A(\Omega) &= \frac{F_0/m}{\sqrt{(\omega_0^2-\Omega^2)^2+(2\beta\Omega)^2}}
\end{align*}

\textbf{Pequeñas oscilaciones desde potencial:}
\begin{align*}
U'(x_0) &= 0 \\
U(x) &\approx U(x_0) + \frac12 U''(x_0)(x-x_0)^2 \\
k_{eq} &= U''(x_0), \qquad
\omega = \sqrt{\frac{U''(x_0)}{m}}, \qquad
T=\frac{2\pi}{\omega}
\end{align*}

\hrulefill

% ==========================================================
\section{MECÁNICA LAGRANGIANA Y HAMILTONIANA}

\subsection*{GLOSARIO DE VARIABLES}
\begin{itemize}[noitemsep]
    \item $\mathcal L$ = lagrangiano
    \item $T$ = energía cinética, $U$ o $V$ = energía potencial
    \item $q_i,\dot q_i$ = coordenadas/velocidades generalizadas
    \item $p_i$ = momento conjugado
    \item $H$ = hamiltoniano
    \item $r,\theta,\varphi$ = coordenadas esféricas
    \item $L_z$ = componente $z$ del momento angular
    \item $v$ = rapidez, $R$ = radio, $\omega$ = velocidad angular
    \item $dl$ = diferencial de camino
\end{itemize}

\subsection*{FÓRMULAS}

\textbf{Lagrangiano:}
\begin{align*}
\mathcal L &= T-U \qquad (\text{equiv. } L=T-V)
\end{align*}

\textbf{Ecuaciones de Euler--Lagrange:}
\begin{align*}
\frac{d}{dt}\left(\frac{\partial \mathcal L}{\partial \dot q_i}\right) - \frac{\partial \mathcal L}{\partial q_i} &= 0
\end{align*}

\textbf{Energía cinética en coordenadas esféricas:}
\begin{align*}
T &= \frac12 m\left(\dot r^2 + r^2\dot\theta^2 + r^2\sin^2\theta\,\dot\varphi^2\right)
\end{align*}

\textbf{Coordenada cíclica:}
Si $\partial\mathcal L/\partial q=0$,
\begin{align*}
p_q &= \frac{\partial\mathcal L}{\partial \dot q} = \text{cte}
\end{align*}
Ejemplo:
\begin{align*}
p_\varphi &= m r^2\sin^2\theta\,\dot\varphi = L_z
\end{align*}

\textbf{Momento conjugado:}
\begin{align*}
p_i &= \frac{\partial\mathcal L}{\partial \dot q_i}
\end{align*}

\textbf{Hamiltoniano:}
\begin{align*}
H &= \sum_i p_i\dot q_i - \mathcal L
\end{align*}
Si conservativo y vínculos fijos:
\begin{align*}
H &= T+V
\end{align*}

\textbf{Ecuaciones de Hamilton:}
\begin{align*}
\dot q_i &= \frac{\partial H}{\partial p_i}, \qquad
\dot p_i = -\frac{\partial H}{\partial q_i}
\end{align*}

\textbf{Rodadura sin deslizamiento:}
\begin{align*}
v &= \omega R
\end{align*}

\textbf{Braquistócrona (funcional):}
\begin{align*}
t &= \int_A^B dt = \int_A^B \frac{dl}{v} \\
dl &= \sqrt{dx^2 + dy^2}
\end{align*}

\end{document}