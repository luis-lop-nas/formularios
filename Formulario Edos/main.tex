\documentclass[10pt,a4paper]{article}

\usepackage[utf8]{inputenc}
\usepackage[spanish]{babel}
\usepackage{amsmath}
\usepackage{amsfonts}
\usepackage{amssymb}
\usepackage{geometry}
\usepackage{enumitem}
\usepackage{titlesec}
\geometry{margin=2cm}

\titleformat{\section}{\Large\bfseries}{\thesection}{1em}{}
\titleformat{\subsection}{\large\bfseries}{\thesubsection}{1em}{}

\setlength{\parindent}{0pt}
\setlength{\parskip}{0.3em}

\begin{document}

\begin{center}
\textbf{\LARGE FORMULARIO DE ECUACIONES DIFERENCIALES}
\end{center}

\hrulefill

% ==========================================================
\section{EDO DE PRIMER ORDEN}

% --------------------------
\subsection*{1) Variables separables}

\textbf{Forma típica:}
\[
y' = f(x)\,g(y)
\]

\textbf{Cambio / propuesta:} separar variables.

\textbf{Procedimiento:}
\begin{enumerate}[noitemsep]
  \item \(\dfrac{dy}{g(y)} = f(x)\,dx\)
  \item Integrar: \(\displaystyle \int \frac{dy}{g(y)}=\int f(x)\,dx + C\)
  \item Aplicar condición inicial si existe.
\end{enumerate}

\textbf{Ejemplo de forma:} \(y'=-\sin x\,y^2\).

\hrulefill

% --------------------------
\subsection*{2) Lineal de 1\textsuperscript{er} orden (factor integrante)}

\textbf{Forma típica:}
\[
y' + p(x)y = q(x)
\]

\textbf{Cambio / propuesta:} factor integrante
\[
\mu(x)=e^{\int p(x)\,dx}.
\]

\textbf{Procedimiento:}
\begin{enumerate}[noitemsep]
  \item Multiplicar por \(\mu\): \(\mu y' + \mu p\,y = \mu q\)
  \item Reconocer: \((\mu y)'=\mu q\)
  \item Integrar: \(\mu y=\int \mu q\,dx + C\)
  \item Despejar:
  \[
  y=\frac{1}{\mu(x)}\left(\int \mu(x)\,q(x)\,dx + C\right).
  \]
\end{enumerate}

\textbf{Ejemplo de forma:} \(y'+2y=e^{-x}\).

\hrulefill

% --------------------------
\subsection*{3) Bernoulli}

\textbf{Forma típica:}
\[
y'+p(x)y=q(x)y^n,\qquad n\neq 0,1
\]

\textbf{Cambio:}
\[
u=y^{\,1-n}\quad \Rightarrow \quad u'=(1-n)y^{-n}y'.
\]

\textbf{Procedimiento:}
\begin{enumerate}[noitemsep]
  \item Multiplicar por \(y^{-n}\) y sustituir \(u\).
  \item Queda lineal:
  \[
  u' + (1-n)p(x)\,u = (1-n)q(x).
  \]
  \item Resolver con factor integrante y volver a \(y=u^{1/(1-n)}\).
\end{enumerate}

\textbf{Ejemplo de forma:} \(y'+2xy=2xy^2\).

\hrulefill

% --------------------------
\subsection*{4) Homogénea de 1\textsuperscript{er} orden (sustitución \(y=vx\))}

\textbf{Forma típica:}
\[
y'=F\!\left(\frac{y}{x}\right)
\]

\textbf{Cambio:}
\[
y=vx \quad\Rightarrow\quad y'=v+xv'.
\]

\textbf{Procedimiento:}
\begin{enumerate}[noitemsep]
  \item Sustituir: \(v+xv'=F(v)\)
  \item Separar:
  \[
  \frac{dv}{F(v)-v}=\frac{dx}{x}
  \]
  \item Integrar y volver a \(y=vx\).
\end{enumerate}

\textbf{Ejemplo de forma:} \(x y' = y + x\cos^2(y/x)\).

\hrulefill

% --------------------------
\subsection*{5) Exacta}

\textbf{Forma típica:}
\[
M(x,y)\,dx+N(x,y)\,dy=0
\]

\textbf{Criterio de exactitud:}
\[
\frac{\partial M}{\partial y}=\frac{\partial N}{\partial x}.
\]

\textbf{Procedimiento:}
\begin{enumerate}[noitemsep]
  \item Hallar \(\Phi\) tal que \(d\Phi=Mdx+Ndy\).
  \item \(\displaystyle \Phi(x,y)=\int M(x,y)\,dx+g(y)\).
  \item Derivar respecto a \(y\), igualar a \(N\) y obtener \(g(y)\).
  \item Solución implícita: \(\Phi(x,y)=C\).
\end{enumerate}

\textbf{Ejemplo de forma:} \((x+\sin x+\cos x)\,dx+\cos y\,dy=0\).

\hrulefill

% --------------------------
\subsection*{6) No exacta con factor integrante \(\mu(x)\) o \(\mu(y)\)}

\textbf{Forma:}
\[
M(x,y)\,dx+N(x,y)\,dy=0
\quad \text{(no exacta)}.
\]

\textbf{Criterios:}

Si
\[
\frac{1}{N}\left(\frac{\partial M}{\partial y}-\frac{\partial N}{\partial x}\right)=f(x),
\quad \Rightarrow \quad
\mu(x)=e^{\int f(x)\,dx}.
\]

Si
\[
\frac{1}{M}\left(\frac{\partial N}{\partial x}-\frac{\partial M}{\partial y}\right)=g(y),
\quad \Rightarrow \quad
\mu(y)=e^{\int g(y)\,dy}.
\]

\textbf{Procedimiento:}
\begin{enumerate}[noitemsep]
  \item Detectar si el criterio depende solo de \(x\) o solo de \(y\).
  \item Multiplicar toda la ecuación por \(\mu\) y convertirla en exacta.
  \item Resolver con el método del potencial \(\Phi\).
\end{enumerate}

\hrulefill

% --------------------------
\subsection*{7) Dependencia \(x\pm y\) (sustitución \(u=x\pm y\))}

\textbf{Forma típica:} aparece \(x-y\) o \(x+y\), p.ej. \(y'=F(x-y)\).

\textbf{Cambio (caso \(u=x-y\)):}
\[
u=x-y \quad\Rightarrow\quad u'=1-y'.
\]

\textbf{Procedimiento:}
\begin{enumerate}[noitemsep]
  \item Sustituir \(y'=1-u'\) en la EDO.
  \item Resolver la EDO resultante (normalmente separable/lineal) para \(u(x)\).
  \item Volver a \(y=x-u\) (o \(y=u-x\) según el caso).
\end{enumerate}

\textbf{Ejemplo de forma:} \(y'=\sin(x-y)\).

\hrulefill

% --------------------------
\subsection*{8) Sustitución \(u=\ln y\)}

\textbf{Forma típica:}
\[
y'=\frac{y\ln y}{x}
\quad \text{o} \quad
y\ln y\,dx+x\,dy=0.
\]

\textbf{Cambio:}
\[
u=\ln y \quad\Rightarrow\quad y'=yu'.
\]

\textbf{Procedimiento:}
\begin{enumerate}[noitemsep]
  \item Sustituir \(y'=yu'\).
  \item Resolver la EDO en \(u(x)\) (separable/lineal).
  \item Volver: \(y=e^u\).
\end{enumerate}

\hrulefill

% --------------------------
\subsection*{9) Riccati}

\textbf{Forma típica:}
\[
y'=a(x)y^2+b(x)y+c(x).
\]

\textbf{Cambio (si se conoce una particular \(y_1\)):}
\[
y=y_1+\frac{1}{v}.
\]

\textbf{Procedimiento:}
\begin{enumerate}[noitemsep]
  \item Sustituir \(y=y_1+\frac{1}{v}\) y simplificar.
  \item Queda lineal en \(v\):
  \[
  v'+(2a y_1+b)v+a=0.
  \]
  \item Resolver con factor integrante y volver a \(y\).
\end{enumerate}

\hrulefill

% --------------------------
\subsection*{10) Clairaut y Lagrange}

\textbf{Idea común:} definir \(p=y'\).

\subsubsection*{10A) Clairaut}

\textbf{Forma:}
\[
y=xp+f(p),\qquad p=y'.
\]

\textbf{Procedimiento:}
\begin{enumerate}[noitemsep]
  \item Derivar:
  \[
  p=p+xp'+f'(p)p' \Rightarrow (x+f'(p))p'=0.
  \]
  \item Familia: \(p=C\Rightarrow y=Cx+f(C)\).
  \item Singular: \(x=-f'(p)\) y \(y=xp+f(p)\).
\end{enumerate}

\textbf{Ejemplo de forma:} \(y=xy'+(y')^2\).

\subsubsection*{10B) Lagrange}

\textbf{Forma:}
\[
y=x\phi(p)+\psi(p),\qquad p=y'.
\]

\textbf{Procedimiento (resumen):}
\begin{enumerate}[noitemsep]
  \item Derivar y obtener una EDO lineal para \(x(p)\).
  \item Resolver \(x(p)\).
  \item Recuperar \(y=x\phi(p)+\psi(p)\).
\end{enumerate}

\hrulefill

% ==========================================================
\section{EDO DE ORDEN 2 (Y SUPERIOR)}

\subsection*{11) Lineal homogénea con coeficientes constantes}

\textbf{Forma:}
\[
y''+ay'+by=0.
\]

\textbf{Propuesta:} \(y=e^{rx}\).

\textbf{Procedimiento:}
\begin{enumerate}[noitemsep]
  \item Característica: \(r^2+ar+b=0\).
  \item Según raíces:
  \begin{itemize}[noitemsep]
    \item \(r_1\neq r_2:\; y=C_1e^{r_1x}+C_2e^{r_2x}\).
    \item doble \(r:\; y=(C_1+C_2x)e^{rx}\).
    \item \(\alpha\pm i\beta:\; y=e^{\alpha x}(C_1\cos\beta x+C_2\sin\beta x)\).
  \end{itemize}
\end{enumerate}

\hrulefill

\subsection*{12) Lineal no homogénea con coeficientes constantes}

\textbf{Forma:}
\[
y''+ay'+by=g(x).
\]

\textbf{Métodos:} coeficientes indeterminados o variación de parámetros.

\textbf{Procedimiento (indeterminados):}
\begin{enumerate}[noitemsep]
  \item Hallar \(y_h\).
  \item Proponer \(y_p\) según la forma de \(g(x)\).
  \item Si hay resonancia, multiplicar por \(x\) lo necesario.
  \item Solución: \(y=y_h+y_p\).
\end{enumerate}

\hrulefill

\subsection*{13) Cauchy--Euler (equidimensional)}

\textbf{Forma:}
\[
x^2y''+a x y'+b y=0.
\]

\textbf{Propuesta:} \(y=x^m\) (con \(x\neq 0\)).

\textbf{Procedimiento:}
\begin{enumerate}[noitemsep]
  \item Sustituir \(y=x^m\) y obtener ecuación indicial en \(m\).
  \item Resolver como una característica (raíces distintas/doble/complejas).
  \item Alternativa: \(x=e^t\) reduce a coeficientes constantes.
\end{enumerate}

\hrulefill

% ==========================================================
\section{MÉTODOS DE SERIES}

\subsection*{14) Series de potencias (punto ordinario)}

\textbf{Forma:}
\[
y''+P(x)y'+Q(x)y=0
\quad \text{con }P,Q\text{ analíticas en }x_0.
\]

\textbf{Propuesta:}
\[
y=\sum_{n=0}^\infty a_n (x-x_0)^n.
\]

\textbf{Procedimiento:}
\begin{enumerate}[noitemsep]
  \item Calcular \(y',y''\) en serie.
  \item Sustituir, agrupar potencias y obtener recurrencia.
  \item \(a_0,a_1\) libres \(\Rightarrow\) dos soluciones LI.
\end{enumerate}

\hrulefill

\subsection*{15) Frobenius (punto singular regular)}

\textbf{Forma:}
\[
x^2y''+x p(x)y'+q(x)y=0
\quad \text{con }p,q\text{ analíticas en }0.
\]

\textbf{Propuesta:}
\[
y=\sum_{n=0}^\infty a_n x^{n+r}.
\]

\textbf{Procedimiento:}
\begin{enumerate}[noitemsep]
  \item Sustituir y obtener ecuación indicial (para \(r\)).
  \item Hallar \(r_1,r_2\) y la recurrencia para \(a_n\).
  \item Si \(r_1-r_2\in\mathbb{Z}\) o coinciden \(\Rightarrow\) puede aparecer \(\ln x\).
\end{enumerate}

\hrulefill

\subsection*{16) Ecuaciones especiales por series}

\textbf{Airy:}
\[
y''\pm xy=0.
\]

\textbf{Chebyshev:}
\[
(1-x^2)y''-xy'+p^2y=0.
\]

\textbf{Truco tipo Hermite (dado):}
Si
\[
y''+\left(p+\frac12-\frac{x^2}{4}\right)y=0,
\quad \text{usar}\quad
y=w(x)e^{-x^2/4}.
\]

\hrulefill

% ==========================================================
\section{SISTEMAS DE EDO}

\subsection*{17) Sistemas lineales homogéneos con matriz constante}

\textbf{Forma:}
\[
\dot{\mathbf x}=A\mathbf x.
\]

\textbf{Propuesta:} \(\mathbf x(t)=e^{At}\mathbf x(0)\) o combinación de \(e^{\lambda t}\mathbf v\).

\textbf{Procedimiento:}
\begin{enumerate}[noitemsep]
  \item Calcular autovalores \(\lambda\) y autovectores \(\mathbf v\).
  \item Si diagonalizable: \(A=PDP^{-1}\Rightarrow e^{At}=Pe^{Dt}P^{-1}\).
  \item \(\mathbf x(t)=\sum_k C_k e^{\lambda_k t}\mathbf v_k\).
\end{enumerate}

\hrulefill

\subsection*{18) Sistemas lineales no homogéneos}

\textbf{Forma:}
\[
\dot{\mathbf x}=A\mathbf x+\mathbf b(t).
\]

\textbf{Variación de constantes:}
\[
\mathbf x(t)=e^{At}\left(\mathbf c+\int e^{-At}\mathbf b(t)\,dt\right).
\]

\hrulefill

\subsection*{19) Sistemas no lineales: puntos críticos y linealización}

\textbf{Forma:} \(\dot x=f(x,y)\), \(\dot y=g(x,y)\).

\textbf{Procedimiento:}
\begin{enumerate}[noitemsep]
  \item Puntos críticos: \(f=0\), \(g=0\).
  \item Jacobiano:
  \[
  J=
  \begin{pmatrix}
  \partial f/\partial x & \partial f/\partial y\\
  \partial g/\partial x & \partial g/\partial y
  \end{pmatrix}.
  \]
  \item Linealizar: \(\dot{\mathbf u}=J(\mathbf u_0)\mathbf u\).
  \item Clasificar por autovalores (nodo, silla, foco, centro).
\end{enumerate}

\hrulefill

\subsection*{20) Estabilidad 2D lineal (traza/determinante)}

Para \(A=\begin{pmatrix}a&b\\c&d\end{pmatrix}\):

\textbf{Asintóticamente estable} si
\[
\operatorname{tr}(A)=a+d<0,
\qquad
\det(A)=ad-bc>0.
\]

\end{document}